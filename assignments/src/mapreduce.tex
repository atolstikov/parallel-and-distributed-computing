
\section{Реализация распределенного алгоритма с использованием MapReduce}

\textit{Работать над лабораторной можно в кооперации двух студентов.}

Для выполнения работы нужно написать программный код, для запуска через Hadoop MapReduce.
\textbf{Кроме самого решения задания нужно написать генератор данных, на которых ваше решение будет запускаться.}

Для сдачи работы рекомендуется попробовать использовать использовать образ с
\href{https://hub.docker.com/r/sequenceiq/hadoop-docker}{сайтаЪ
(можно найти более новый или просто совсем другой образ).
Кроме этого можно самостоятельно развернуть Hadoop окружение и делать задание без виртуалок/образов.
Если не согласовывать отдельно, то выполнять работу необходимо, используя JAVA или нативный стримминг в Hadoop.
В документации вы сможете найти пример с подсчетом частот слов WordCount.
Дополнительно будет оцениваться организация «кластера» и запуск приложения в распределенной среде.
Например, можно использовать \href{https://cloud.yandex.ru/docs/partner/qa/start-grant}{Грант} от Yandex.Cloud.

\subsection	Статистики (1 балл)

Входные данные: набор файлов со статистическими данными.\\
Формат: ключ [space | tab] значение (double | int)\\
Выходные данные: набор файлов со статистиками.\\
Формат: ключ [space | tab] среднее [space | tab] дисперсия

\subsection	Медиана (1-2 балла, в зависимости от реализованного алгоритма)
Входные данные: набор файлов со статистическими данными.
Формат: ключ [space | tab] значение (double | int)
Выходные данные: набор файлов со статистиками.
Формат: ключ [space | tab] медианное значение

\subsection	MR_JOIN (1-2 балла, в зависимости от реализации)
Входные данные: набор файлов со связями между записями.
Формат: 1 [space | tab] столбец А [space | tab] столбец Б
               2 [space | tab] столбец Б [space | tab] столбец В
Выходные данные: результат связывания столбца А и столбца В.
Формат: столбец А [space | tab] столбец В

3.	Обратный индекс (1 балл)
Входные данные: текст статей из википедии (минимум 20).
Формат: первая строка – название статьи, вторая и далее – текст.
Выходные данные: для всех слов, которые встречаются в этих статьях, найти список статей, в которых они встречаются.
Формат: слово [tab] список названий статей разделенных ‘\t’.

4.	Фильтр (1 балл)
Входные данные: текст статей из википедии (минимум 20), файл со списком слов для фильтрации.
Формат: первая строка – название статьи, вторая и далее – текст; для фильтра по одному слову в строке.
Выходные данные: для всех слов, заданных в фильтре, найти список статей, в которых они встречаются.
Формат: слово [tab] список названий статей разделенных ‘\t’.

5.	Гистограммы (1 балл)
Входные данные: набор файлов со статистическими данными, числовой параметр – количество столбцов в гистограммах.
Формат: ключ [space | tab] значение (double из интервала [0, 1])
Выходные данные: набор файлов со статистиками.
Формат: ключ [space | tab] числовая гистограмма, содержащая заданное количество столбцов.

6.	Веб-робот (1-2 балла, в зависимости от реализации)
Входные данные: список стартовых страниц для робота, числовой параметр для глубины захода
Формат: по одной странице в строке
Выходные данные: список страниц, до которых добрался робот
Формат: текстовые файлы, по одному адресу в строке

7.	Странная популярность (1-2 балла, в зависимости от реализации)
Входные данные: скачать несколько архивов https://archive.org/download/google_ngrams-eng-1M-1gram
Формат: https://archive.org/details/google_ngrams-eng-1M-1gram
Выходные данные: Для каждого слова подсчитайте дисперсию его частот. Выведите ТОП-20 слов с максимальной дисперсией.
Формат: слово, дисперсия, все частоты

Предлагайте Ваши варианты возможных заданий по данной теме. Задания, которые будут включены в курс, Вам принесут дополнительный балл. Желательно предлагать задания с массивом данных уже в интернете.

